\documentclass[a4paper]{article}

\usepackage[T1]{fontenc}
\usepackage{graphicx}
\usepackage{amsmath}
\usepackage[utf8]{inputenc}
\usepackage{enumitem}
\setlist[description]{style=unboxed}

\usepackage{tikz}
\usepackage{pgfplots}
\usepackage{circuitikz}
\usepackage{tabularx}
\usepackage{rotating}
\usepackage{caption} 
\captionsetup[table]{skip=10pt}

\usetikzlibrary{calc,positioning,shapes,decorations.pathreplacing}

\tikzset{
	short/.style={draw,rectangle,text height=3pt,text depth=13pt,
		text width=7pt,align=center,fill=gray!30},
	long/.style={short,text width=1.5cm},
	verylong/.style={short,text width=4.5cm}
}

\begin{document}
\section{DC voltage power supply - solutions}

\subsection{Finding the proper choice of components}
\label{ele:task:1}

Determine voltage from output:
\begin{equation}
U_{out} = V_z \cdot (1 + \frac{R_2}{R_4})
\end{equation}

Find nominal current of Zener diode (typically 20 mA). Voltage beyond the 
rectifier is 12 V ($13.2 - 2 \cdot 0.6$). Therefore:
\begin{equation}
R_s = \frac{12 - U_z}{I_z}
\end{equation}

\subsection{LC filter design}
\label{ele:task:2}
Determine system transfer function:
\begin{equation}
H(s) = \frac{1}{LC} \frac{1}{s^2 + \frac{1}{R_LC} s + \frac{1}{LC}}
\end{equation}
Compare with second order filter transfer function:
\begin{equation}
H(s) = \frac{\omega_0^2}{s^2 + \frac{\omega_0}{R_LC} s + \omega_0^2}
\end{equation}
Therefore:
\begin{equation}
Q = \omega_0 R_L C
\end{equation}
For getting $\omega_0$ solve:
\begin{equation}
20 \log |H(j\omega)| = -20 \text{dB}
\end{equation}

\subsection{Side task: Reliability of the system with redundant power supply} 
\label{ele:task:3}

Solution:
\begin{equation}
R(t) = \frac{1}{\lambda_1 - \lambda_2} (\lambda_1 e^{-\lambda_2 t} - 
\lambda_2 e^{-\lambda_2 t})
\end{equation}
For $t = 10000$ h and $\lambda_1 = 1 \cdot10^{-6}$ $h^{-1}$, 
$\lambda_2 = 2 \cdot10^{-6}$ $h^{-1}$ probability of failure is:
\begin{equation}
F(10000) = 1 - R(10000) = 9.9 \cdot 10^{-5}
\end{equation}


\subsection{Grading scheme}
Tasks are graded in the following manner:
\begin{itemize}
\item tasks \ref{ele:task:1} and \ref{ele:task:2} - up to \textbf{10 pts}, 
up to \textbf{5 pts} for each component.
\item task \ref{ele:task:3} - up to \textbf{10 pts} - \textbf{5 pts} for the 
correct probability and up to \textbf{5 pts} for correct documentation.
\end{itemize}

 
\end{document}