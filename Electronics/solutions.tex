\documentclass[a4paper]{article}

\usepackage[T1]{fontenc}
\usepackage{graphicx}
\usepackage{amsmath}
\usepackage[utf8]{inputenc}
\usepackage{enumitem}
\setlist[description]{style=unboxed}

\usepackage{tikz}
\usepackage{pgfplots}
\usepackage{circuitikz}
\usepackage{tabularx}
\usepackage{rotating}
\usepackage{caption} 
\captionsetup[table]{skip=10pt}

\usetikzlibrary{calc,positioning,shapes,decorations.pathreplacing}

\tikzset{
	short/.style={draw,rectangle,text height=3pt,text depth=13pt,
		text width=7pt,align=center,fill=gray!30},
	long/.style={short,text width=1.5cm},
	verylong/.style={short,text width=4.5cm}
}

\begin{document}
\section{DC voltage power supply - solutions}

\subsection{Finding the proper choice of components}
\label{ele:task:1}

Determine vo

\subsection{LC filter design}
\label{ele:task:2}

\subsection{Side task: Reliability of the system with redundant power supply} 
\label{ele:task:3}

\subsection{Grading scheme}
Tasks are graded in the following manner:
\begin{itemize}
\item tasks \ref{ele:task:1} and \ref{ele:task:2} - up to \textbf{10 pts}, 
up to \textbf{5 pts} for each component.
\item task \ref{ele:task:3} - up to \textbf{10 pts} - \textbf{5 pts} for the 
correct probability and up to \textbf{5 pts} for correct documentation.
\end{itemize}

 
\end{document}