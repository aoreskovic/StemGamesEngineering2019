UVOD

Podvodni habitat ostvaruje vezu s brodom pomoću ultrazvučnih hidrofona postavljenih na vrh habitata. Kako bi se povećao domet ultrazvučne komunikacije, više hidrofona se može povezati u beamforming sustav. Za to je na vrhu habitata predviđena pravokutna ravna površina veličine 1 m x 1 m, koja je prikazana u referentnom koordinatnom sustavu na slici \ref{fig:coord}.
\begin{figure}[h!]
   \centering
   \includegraphics[width=0.4\textwidth]{ula.png}
   \caption{Reference coordinate system with available beamformer area}
   \label{fig:coord}
\end{figure}

Hidrofon radi u frekvencijskom rasponu od 20 kHz do 90 kHz, a za komunikaciju je odabrana frekvencija nosioca od 30 kHz. Smatrajte da je signal uskopojasan. Hidrofon ima kosinusni dijagram zračenja opisan formulom
[\ D(\phi,\theta) = \cos^2(\theta) ]\
gdje je $\phi$ kut azimuta, a $\theta$ kut elevacije. Dijagram zračenja u referentnom koordinatnom sustavu je prikazan slikom \ref{fig:hydrophone}.
\begin{figure}[h!]
   \centering
   \includegraphics[width=0.4\textwidth]{hydrophone.png}
   \caption{Farfield of a hydrophone element}
   \label{fig:hydrophone}
\end{figure}
Hidrofoni su fizički kružni, promjera 5 cm. Svi hidrofoni antenskog sustava postavljaju se s maksimumom zračenja u zenit, tj. okomito na površinu mora. S druge strane, brod raspolaže s jednim hidrofonom montiranim na trup broda i okrenutim u nadir, tj. okomito na morsko dno.

Smatrajte da se komunikacija odvija u Jadranskome moru, prosječnog saliniteta 28\% i u temperaturi mora od 20\circ.


TASK 1

Potrebno je dizajnirati linearni antenski niz za postizanje najboljeg linka između dviju stanica. Zadana su tri scenarija:
Scenarij 1
Brod se naalzi točno iznad antenskog sustava.
Scenarij 2
Brod se nalazi iznad uzdužne osi antenskog sustava, na elevaciji od 60 stupnjeva (note to self: u odnosu na horizont)
Scenarij 3
Brod se nalazi iznad uzdužne osi antenskog sustava, na elevaciji od 30 stupnjeva.
Scenarij 4
Brod se nalazi iznad uzdužne osi antenskog sustava, na elevaciji od 10 stupnjeva.

Dizajnirajte jedan linearni antenski niz koji će raditi dobro za sva 4 scenarija. Antenski niz je definiran y koordinatom svakog antenskog elementa. Broj elemenata je proizvoljan, uz ograničenja zadana u uvodnom dijelu zadatka.

Za svaki od 4 scenarija odredite težinske faktore koji se dovode na svaki antenski element kako bi se postigao najbolji mogući beamforming za taj scenarij.


Output data

elementCoordinates - Jednodimenzionalni niz koji označava pozicije antenskog elementa na y osi

beamformingWeights - Jednodimenzionalni niz kompleksnih brojeva duljine jednake elementCoordinates.

Svaki element beamformingWeights niza odgovara težini primijenjenoj na respektivni element iz elementCoordinates niza.


Scoring

Za sve zadane scenarije evaluira se antenski sustav koji ste dizajnirali. U prva tri scenarija, bodovi se dodjeljuju za usmjerenost koju antenski sustav ostvaruje u smjeru broda.
Ako se brod nalazi u glavnoj latici ili u prvoj sporednoj latici, tada tim dobiva bodove za dani scenarij. No, ako se brod nalazi u prvoj sekundarnoj latici, timu se oduzimaju bodovi koji odgovaraju razlici najveće usmjerenosti antenskog niza i usmjerenosti u smjeru broda.

U scenariju 4, dizajnirani antenski sustav se evaluira sa težinskim faktorima koje izračunate, te uspoređuje usmjerenost koju antenski sustav ostvaruje u smjeru broda u odnosu na usmjerenost koju bi ostvario u slučaju uniformne pobude. Najveći broj bodova koji se u ovom scenariju može ostvariti je 10.

Antenski sustav koji se sastoji od samo jednog elementa ne dobiva bodove ni u kojem scenariju.
