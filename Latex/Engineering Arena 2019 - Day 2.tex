\documentclass{article}
\usepackage{standalone}
\graphicspath{{Images/}}
\usepackage[e]{Template/gameshf}
\usepackage{import}
\usepackage{xcolor}

%opening
\title{STEM games 2019}
\author{Mentori}

\begin{document}
	
	\maketitle
		
	\section{Introduction}
	
	The choice of propulsion system type is one of the most important parts in submarine’s design process since it determines the ability of fulfilling its purpose. Submarines that perform underwater research activities require endurance and certain flexibility in different littoral conditions. For example, conventional motor propulsion systems, such as diesel electric, when operational, require fresh air intake for driving electrical generator. In this way, a period while a submarine can remain submerged is limited by the power and endurance of the battery system and counts in hours or few days at most. Therefore, when it comes to endurance such systems might not be an optimal choice. Technology that overcomes this drawback is a nuclear power system which offers better endurance (even few months) and speed of the submarines but there are number of disadvantages (very expensive, operational limitations in shallow littoral waters etc.) that make them unsuitable for underwater research activities. The most suitable seems to be so called Air Independent Propulsion systems (AIP systems) which are marine propulsion technologies that allows non-nuclear submarine to operate without access to atmospheric oxygen. 
	
	One of the most interesting is definitely French AIP technology so called MESMA (Module d'Energie Sous-Marine Autonome). This is the AIP system that uses closed-turbine cycle for powering electrical generator and the heat for steam generation is provided by liquid ethanol (ili bioethanol) combustion in pure oxygen (sometimes mixed with argon). Combustion of pure ethanol (ili bioethanol) makes this system ecologically much more acceptable than other systems that use conventional fuels. In this way, virtually silent MESMA exploits all advantages of AIP systems for maintenance of underwater habitats and performing underwater research or training activities.
	 
	Today's assignment is to design a MESMA power generation system for submarine propulsion for a given submarine hull which consists of outer and inner shell, showed on Figure 1. Inside inner shell there is a reserved space of fixed volume (zadati ovdje ili zajedno sa skicom!) consisting of three compartments; fuel tanks compartment (Compartment 1), engineering room compartment (Compartment 2) and cargo place compartment (Compartment 3). Individual volumes of each compartment can be arbitrarily chosen but the sum of all compartment volumes is fixed to a constant value. 
	
	The scheme of MESMA power generation system is given in the assignment description and its performance (range, speed, efficiency, price?) (dopuniti, ako treba, kada tablica bodovanja bude gotova) will be evaluated for different depths according to the scoring table (možda nije dobar naziv) given later in the text. It is important to emphasize that one of evaluation categories is the size of the cargo place compartment so the care must be taken not to oversize the propulsion system which must be placed in other two compartments.
	
	The assignment can be divided into two steps: 
	
	\begin{enumerate}
		\item Design and dimension main components of the propulsion system,
		\item Define operational characteristics of some components.
	\end{enumerate}

	After the first step you will be given a full model of propulsion system with the components you selected which will help you define operational characteristics of some components.
	Detailed description of the system components is given in the next section.
	
	Engineers, good luck!
	
	Zadatak,
	Your task is to design submarine power generation system which can generate sufficient energy to travel xx km and generate minimum required power during xx hours when submarine is being loaded and unloaded.
	or: to design power generation system which can generate sufficient energy to travel 100 km at depth 50 m 20 km at depth  100 m and 20 km at depth 10 m
	Formule za nosivost, masu, otpor, maksimalna duljina izmjenjivača, maksimalna snaga turbine i drugo ovisno o veličini podmornice
		
		
	\section{Power generation system description}
	
	Power generation system schematics are shown on figure \ref{fig:pwr_scheme}.
	
	\begin{figure}[h!]
		\centering
		\includegraphics[width=\textwidth]{Thermodynamics/pwr_system.png}
		\caption{Power generation system.}
		\label{fig:pwr_scheme}
	\end{figure}
	
	Submarine power generation system consists of two main circuits; flue gas circuit (red) and steam circuit (cyan). Steam circuit is used for steam production which expands in turbine "Turb" providing necessary power for the "Generator" used for charging the batteries for submarine propulsion. The flue gas circuit is used to provide necessary heat flow for evaporating and superheating the steam for the turbine. Flue gas is generated in the combustion chamber "Comb" in which energy is released by ethanol combustion in pure oxygen.
	
	Liquefied oxygen and ethanol fuel are stored in separate tanks "T1" and "T2", respectively. While pump P2 delivers fuel stream "h" to combustion chamber "Comb", cryogenic pump P2 delivers liquefied oxygen stream "i" to oxygen vaporizer "Vap" where it vaporizes and enters combustion chamber "Comb" in stoichiometric ratio at which fuel combustion is, by assumption, complete without any excess oxygen. Flue gas temperature at the outlet from the combustion chamber (stream "a") is controlled by return stream of cooled flue gas "d" by compressor "Comp" and should never exceed maximum allowed flue gas temperature %ϑ_(fg,max).
	
	Flue gas stream "a", which exits combustion chamber "Comb" enters heat exchanger "HE" used for water evaporation and steam superheating. Therefore, heat exchanger "HE" comprises two sections; an evaporation section "Evap" and superheating section "Super". Total flue gas mass flow which enters heat exchanger "HE" is divided into two flue gas streams "b" and "c". Stream "c" flows through evaporation ("Evap") section and stream "b" flows through superheating ("Super") section in parallel configuration. On heat exchanger exit, two flue gas parallel streams "b" and "c" are adiabatically mixed in mixing chamber "Mix" forming flue gas stream "m".Part of stream "m" is returned by compressor "Comp" to combustion chamber "Comb" as stream "d" to prevent overheating (maximum allowed flue gas temperature), while the rest (stream "e") is used for oxygen evaporation in heat exchanger "Vap" and evacuated outside submarine.
	  
	Heat exchanger "HE" is a pool boiling type apparatus which means that liquid water in the evaporating section is at rest and is evaporated by the flue gas that passes through a tube bundle in the evaporating section. The exact amount of steam that is evaporated in evaporation section "Evap" (which corresponds to the steam mass flow of stream "g") is then superheated in superheating section "Super" of heat exchanger "HE" producing the superheated steam "f". Superheated steam (stream "f") passes through throttling valve "TV" and enters the steam turbine "Turb" where mechanical power is produced by steam adiabatic expansion to condensation pressure p2. After the expansion, low pressure steam condenses in sea water cooled condenser "Cond" and is returned to high pressure evaporator "Evap" by condensate pump "P3". 
	
	The turbine drives an alternator-rectifier which supplies submarine propulsion system with direct current. (?) Required mechanical power for submarine propulsion and auxiliary systems should in every timestep be provided by turbine (no buffer). (dogovor s mentorskim teamom)
	
	Following sections provide a detailed description of every part of power generation system that should be modeled.
	
	\section{Turbine}
	
	A Steam Turbine is a mechanical device in which thermal energy is extracted from pressurized steam by its adiabatic expansion and is transformed to mechanical work.  Therefore, for a given mass flow rate of steam qm, power $P$ generated by steam turbine is defined:
	
	\begin{equation}\label{eq:power}
		P = q_m(h_1 - h_2),
	\end{equation}
	
	\noindent
	where 
	
	\begin{itemize}
		\item $q_m$ is steam mass flow rate, 
		\item $h_1$ is steam specific enthalpy at turbine inlet,
		\item $h_2$ is steam specific enthalpy at turbine outlet.
	\end{itemize}

	Turbine steam swallowing capacity mass flow rate $q_m$ at given pressures and inlet temperature can be expressed by Stodola’s Ellipse equation:
	
	\begin{equation}\label{eq:stodola}
		q_m = \frac{K}{\sqrt{T}}(p_1^2 - p_2^2)^\frac{1}{2},
	\end{equation}
	
	\noindent
	where:
	
	\begin{itemize}
		\item $K$ is Stodola’s coefficient, 
		\item $p_1$ is turbine inlet pressure,
		\item $p_2$ is turbine outlet pressure,
		\item $T$ is turbine inlet absolute temperature.
	\end{itemize}
	
	\section{Steam turbine isentropic efficiency}
	
	Turbine isentropic efficiency $\eta$ is defined as the ratio between enthalpy drop at actual turbine expansion and enthalpy drop at isentropic expansion between inlet and outlet turbine pressure $p_1$ and $p_2$.
	
	\begin{equation}\label{eq:eta}
		\eta = \frac{h_1-h_2}{h_1 - h_{2s}}.
	\end{equation}
	
	Steam turbine isentropic efficiency depends upon mean steam volume flow rate qv according to the following equation:
	
	\begin{equation}\label{eq:eta2}
		\eta = aq_v^2 + bq_v + c,
	\end{equation}
	
	\noindent
	where $q_v$ is geometric mean of volume flow rates at turbine inlet ($q_{v,in}$) and at outlet in case of isentropic expansion ($q_{v,out,s}$). 
	
	\begin{equation}\label{eq:q_v}
		q_v = \sqrt{q_{v,in} \cdot q_{v,out,s}},
	\end{equation}
	
	Turbine efficiency is highest at its nominal design point for which pressures ($p_1$, $p_2$) and inlet temperature T1 are given in table 1, while nominal flow rate is defined by Stodola’s ellipse equation \ref{eq:stodola}. Turbine can operate at other conditions (off design turbine performance with lower, i.e. higher mass flow rates than the nominal) with the cost of reduced efficiency, provided that these conditions are not outside permitted values.
	
	\section{Off design turbine performance}
	
	\subsection*{Lower mass flow rate}
	
	Steam throttling is a common way for reduction of power produced by steam turbine. By reducing fuel pump “P2" and oxygen pump “P1" speed accordingly when power demand is lowered, fuel consumption and evaporated steam mass flow rate are decreased. In conjunction with steam flow rate decrease, throttling valve “TV" orifice is constricted to reduce turbine inlet pressure $p_{1,thr}$ to a value which satisfies Stodola’s Ellipse law \ref{eq:stodola} for given decreased mass flow rate $q_m$. In a throttling process, superheated steam which exits heat exchanger “HE" is throttled to lower pressure $p_{1,thr}$, while steam enthalpy is conserved.
	
	\begin{equation}\label{eq:ent_preserve1}
		h_{1,thr} = h_1,
	\end{equation}
	\begin{equation}\label{eq:ent_preserve2}
	p_{1,thr} = p_1,
	\end{equation}
	\begin{equation}\label{eq:ent_preserve3}
	T_{1,thr} = T(p_{1,thr},h_1) < T_1.
	\end{equation}
	
	Side effect of throttling is a temperature reduction which corresponds to the state of conserved enthalpy $h_1$ at throttled pressure $p_{1,thr}$. Pressure and temperature are both part of a Stodola’s Ellipse law \ref{eq:stodola} which means they need to simultaneously satisfy Stodola’s Ellipse law  \ref{eq:stodola}  and enthalpy conservation of a throttling process from state "1".
	
	By throttling, turbine inlet pressure and inlet temperature are reduced which decreases available isentropic power of turbine. Turbine efficiency described by equation \ref{eq:eta2} is also decreased at partial turbine load.
	
	\subsection*{Higher mass flow rate}
	
	In case that power demand is increased, turbine can be overloaded by steam mass flow $q_m$ greater than nominal. This is achieved by fuel pump "P2" and oxygen pump "P1" speed increase which causes increased fuel consumption and evaporated steam mass flow. When turbine is overloaded, steam flow bypasses few initial turbine stages which enables increased steam consumption, although with reduced turbine efficiency described by equation (4). To accurately model this effect is out of scope of this assignment. Therefore, it is assumed in overload working mode that turbine inlet values of pressure and temperature remain nominal, i.e. there is no change at throttling valve due to increased steam mass flow rate $q_m$. The only effect of increased $q_m$ is turbine efficiency reduction as described by equation \ref{eq:eta2}.
	
	Restrictions of steam turbine operational parameters are following:
	
	\begin{itemize}
		\item Steam temperature at turbine inlet should not exceed temperature $\vartheta_{turb,max}$,
		\item Define operational characteristics of some components.
	\end{itemize}
	
	Turbine design parameters are given in \ref{tab:turb_params}.
	
	\section{Heat exchangers}
	
	Heat exchanger is a device whose purpose is to transfer heat from warmer fluid stream to colder stream. There are three heat exchangers in the power generation system: heat exchanger “HE" which is divided to superheating and evaporative section, condenser "Cond" and oxygen vaporizer "Vap". Condenser "Cond" and vaporizer "Vap", whose performance will not be evaluated in this task are assumed to be regulated accordingly in each time step with the current need of the power generation system. 
	
	\subsection*{Heat exchanger "HE"}
	
	Heat exchanger "HE" is used for steam evaporation and superheating in two corresponding sections by hot flue gas from combustion chamber ("b" and "c" stream). It is shell and tube type with flue gas passing through tubes in each section. 
	
	Heat flow exchanged in each of two heat exchanger sections is described as:
	
	\begin{equation}\label{eq:heat_flow}
		\Phi = A \cdot k \cdot LMTD,
	\end{equation}
	
	where:
	
	\noindent
	$LMTD$ is logarithmic mean temperature difference, 
	$k$ is overall heat transfer coefficient,
	$A$ is total heat exchange area of the section. 
	Assume superheating section of heat exchanger is counter-flow.
	
	In addition to LMTD method, rate of heat transfer can be determined using NTU method. For counter-flow heat exchanger effectiveness is calculated as:
	
	\begin{equation}\label{eq:heat_exchanger_eff}
		\pi_1 = \frac{1 - exp((1-\pi_3)\pi_2)}{1-\pi_3exp(-(1-\pi_3)\pi_2)},
	\end{equation}
	
	Dimensionless parameters $\pi_1$,$\pi_2$,$\pi_3$ are defined as follows:
	
	\begin{equation}\label{eq:pi_params}
		\pi_1 = \frac{\theta'_1 - \theta''_1}{\theta'_1 - \theta'_1},
	\end{equation}
	
	\begin{equation}\label{eq:pi_params2}
		\pi_2 = \frac{kA}{C_1},
	\end{equation}
	
	\begin{equation}\label{eq:pi_params3}
		\pi_3 = \frac{C_1}{C_2},
	\end{equation}
	
	\noindent
	where $C$ is stream heat capacity.
	
	Index 1 corresponds to stream with lower heat capacity, index 2 to stream with higher heat capacity, superscript $'$ to stream inlet state in heat exchanger and $''$ to outlet state.
	
	Flue gas, which is divided into two streams "b" and "c" enters evaporation "Evap" and superheating "Super" section at equal temperature and flows through tubes in one pass. Flue gas mass flow through evaporative and superheating section of heat exchanger is defined as:
	
	\begin{equation}\label{eq:flue_gas_flow1}
		q_{m,fg,evap} = evap_{fr} \cdot q_{m,fg},
	\end{equation}
	
	\begin{equation}\label{eq:flue_gas_flow2}
		q_{m,fg,super} = (1-evap_{fr}) \cdot q_{m,fg},
	\end{equation}
	
	\noindent
	Where:
	$evap_{fr}$ is flue gas fraction of stream "c" to total flue gas stream "a"  total flue gas mas flow and is assumed constant throughout system operation. Its value can be selected between 0 and 1.
	$q_{m,fg}$ is total flue gas mass flow rate (stream "a") .
	$q_{m,fg,evap}$ is evaporation section mass flow rate (stream "b")
	$q_{m,fg,super}$ is superheating section mass flow rate (stream "c")
	Heat transfer coefficient on tube side can be calculated by using following simplified convection models.
	
	\noindent
	If the flow is transient or turbulent ($2300 < Re < 5e6$) , mean Nusselt number is defined as:
	
	\begin{equation}\label{eq:nusselt}
		Nu = \frac{f/8 \cdot (Re - 1000) \cdot Pr}{1+12.7\sqrt{f/8} \cdot (Pr^{2/3}-1)},
	\end{equation}
	
	\noindent
	where $f$ is the friction factor, defined as
	
	\begin{equation}\label{eq:fric_factor}
		f = \frac{1}{(1.82\log_{10}Re - 1.64)^2}.
	\end{equation}
	
	\noindent
	If the flow is laminar ($Re<2300$), mean Nusselt number is defined as:
	
	\begin{equation}\label{eq:nusselt2}
		Nu = 1.86 \left(Pe \frac{d}{L}\right)^{1/3}.
	\end{equation}
	
	\noindent
	In above equations non-dimensional numbers are
	
	\begin{itemize}
		\item $Nu = \frac{\alpha d}{\lambda}$ Nusselt number
		\item $Re = \frac{\rho w d}{\mu}$ Reynolds number
		\item $Pr = \frac{\mu c_p}{\lambda}$ Prandtl number
		\item $Pe = Re Pr$ Peclet number
	\end{itemize}

	\noindent
	Nusselt, Prandtl and Reynolds numbers and flue gas velocity are calculated using mean flue gas temperature properties between pipe inlet and outlet. All physical properties except density are assumed independent of pressure and its values for pure gases are given in appendix. Flue gas physical properties are calculated as molar average of gases which form flue gas. Density is calculated using ideal gas equation of state.
	
	\noindent
	Evaporation occurs on shell side of evaporative section of heat exchanger. Evaporative heat transfer coefficient on shell side is assumed constant and its value is $\alpha_{ev}=3500 \frac{W}{m^2 K}$. In superheater section of exchanger, heat transfer coefficient on shell side is also assumed constant and its value is $\alpha_{super}=70 \frac{W}{m^2 K}$.
	
	Evaporated steam mass flow $q_m$ is determined by heat flow $\Phi_e$ exchanged in evaporative part "Evap" of heat exchanger "HE":
	
	\begin{equation}\label{eq:evap_steam_mass}
		\Phi_e = q_m(h'' -h_4),
	\end{equation}
	
	where $h''$ is saturated (dry) steam specific enthalpy, and h4 is enthalpy of subcooled liquid at condensate pump outlet. 
	The same amount of heat flow is received from stream "b" flue gas whose enthalpy decreases.
	
	\begin{equation}\label{eq:evap_steam_mass2}
		\Phi_e = q_{m,fg,evap} (h_{fg,comb} - h_{fg,evap,out}),
	\end{equation}
	
	\noindent
	where 
	$h_{fg,comb}$ is flue gas specific enthalpy at combustion chamber "Comb" outlet
	$h_{fg,evap,out}$ is flue gas specific enthalpy at evaporative section "Evap" outlet.
	Flue gas mean molar heat capacities are given in appendix
	It is assumed that mass flow $q_m$ in current time step flows through superheating section "Super" of "HE", turbine "Turb", condenser "Cond" and condensate pump "P3".
	
	Evaporated steam mass flow $q_m$, whose value is determined by equation (13) is then superheated at constant pressure in superheater section “Super" of heat exchanger "HE" by exchanged heat flow:
	
	\begin{equation}\label{eq:evap_steam_mass3}
		\Phi_s = q_m(h_1 - h''),
	\end{equation}
	
	The same amount of heat flow is received from stream "c" flue gas whose enthalpy decreases.
	
	\begin{equation}\label{eq:evap_steam_mass4}
		\Phi_s = q_{m,fg,super} (h_{fg,comb} - h_{fg,super,out}),
	\end{equation}
	
	\noindent
	where $h_{super,out}$ is flue gas specific enthalpy at superheating section "Super" exit. Both flue gas streams are adiabatically mixed at heat exchanger outlet in mixing chamber "Mix". Outlet temperature from the mixing chamber is defined by equation:
	
	\begin{equation}\label{eq:outlet_temp}
		\Theta_{fg,ret} = \frac{q_{m,fg,evap} \cdot h_{fg,evap,out} + q_{m,fg,super} \cdot h_{fg,super,out}}{q_{m,fg} \cdot c_{p,fg} (\Theta_{fg,ret})},
	\end{equation}
	
	\noindent
	where:
	
	\begin{itemize}
		\item $q_{m,fg}$ is temperature after stream "b" and "c" mixing at which part of flue gas is returned to combustion chamber
		\item $c_{p,fg} (\Theta_{fg,ret})$ is mean specific heat capacity of flue gas between temperature $\Theta_{fg,ret}$ and $0^{\circ}C$.
	\end{itemize}
	
	\noindent
	\textcolor{red}{Data for available tubes are given in table 2.}
	
	\noindent
	Value of exchanged heat at both sections depends upon stream inlet conditions, overall heat transfer coefficient k and overall heat transfer area A. Heat transfer coefficient k and area A are influenced by number of tubes in each section and its length which define tube side heat exchanger coefficient and total heat transfer area.
	
	Type and length of tubes at evaporative and superheating section are equal. Number of tubes for each section and flue gas fraction which enters each section can be selected independently.  
	
	While designing a heat exchanger, to prevent tube erosion, one must consider that product of flue gas density, $\rho$, and squared velocity, $w^2$, never exceed $6800 Pa$, i.e. 
	
	\begin{equation}\label{eq:flue_gas_density}
		\rho w^2 < 6800 Pa.
	\end{equation}
	
	\subsection*{Condenser}
	
	\noindent
	Condenser "Cond" is a heat exchanger which is used for condensation of steam which exits turbine "Turb" at condensation pressure $p_2$. During condensation, following heat flow is released:
	
	\begin{equation}\label{eq:heat_flow_cond}
		\Phi_{cond} = q_m (h_2 - h'),
	\end{equation}
	
	\noindent
	where $h'$ is enthalpy of saturated liquid water at condensation pressure $p_2$.
	
	\noindent
	Heat flow $\Phi_{cond}$ is rejected to sea water inside condenser “Cond". It can be assumed that condenser "Cond" in the system is properly designed and sea water mass flow at pump "P4" is automatically adjusted to absorb heat flow $\Phi_{cond}$. Therefore, sizing of condenser "Cond" and sea water pump "P4" and its control is not a part of this task and is assumed as ideal. Condensed water always exits condenser “Cond" at saturated liquid state.
	Maximum sea water temperature at which submarine has to operate is $30^{\circ}C$, and minimal temperature difference between condensing steam and sea water for given condenser is $15^{\circ}C$. 
	
	\subsection*{Combustion chamber}
	
	Combustion chamber "Comb" is enclosed space in which ethanol burns in liquid oxygen. It is assumed to be perfectly insulated. During the combustion, lower heating value of ethanol is converted into flue gas enthalpy. Combustion pressure is at least 1 bar above static pressure evaluated at submarine centerline. Pressure loss from combustion chamber "Comb" to heat exchanger "HE" can be neglected.
	
	In addition to ethanol combustion, combustion chamber "Comb" is used for mixing newly formed flue gas and cooled flow gas which passed heat exchanger "HE" and is returned to chamber "Comb" to prevent overheating. Combustion chamber inlet streams are pure oxygen stream "i" and ethanol stream "h", both at $10^{\circ}C$ and returning flue gas stream "d" at heat exchanger exit temperature  $\Theta_{fg,ret}$. The only outlet stream from the combustion chamber "Comb" is flue gas stream "a" which consists of return flue gas "d" and flue gas newly formed by combustion. 
	
	\begin{equation}\label{eq:flue_gas_stream_new}
		q_{m,fg} = q_{m,fg,new} + q_{m,fg,ret},
	\end{equation}
	
	\noindent
	Flue gas outlet temperature is defined by first law of thermodynamics applied to combustion chamber "Comb":
	
	\begin{equation}\label{eq:flue_gas_out_temp}
		\Theta_{fg,comb} = \frac{\dot{H}_{O2,in} + \dot{H}_{ethanol,in} +\dot{H}_{fg,ret} + q_{n,ethanol} \Delta H_{md}}{C_{fg}},
	\end{equation}
	
	\noindent
	where
	
	\begin{itemize}
		\item $\Delta H_{md}$ is ethanol lower heating value [kJ/kmol]
		\item $\dot{H}_{O2,in}$ is enthalpy of oxygen stream "i" at $10^\circ C$
		\item $\dot{H}_{ethanol,in}$ is enthalpy of ethanol stream "h" at $10^\circ C$
		\item $\dot{H}_{fg,ret}$ is enthalpy of returning flue gas at temperature $\Theta_{fg,ret}$
		\item $q_{n,ethanol}$ is molar flow rate of ethanol stream "h"
		\item $C_{fg}$ is mean isobaric heat capacity of flue gas stream "a" which exits combustion chamber "Comb"
	\end{itemize}

	\noindent
	Flue gas composition is the same in all system parts and is defined by complete combustion of ethanol in pure oxygen.
	
	\section{Fuel and oxygen pumps}
	
	Fuel pump “P2" is used to transport fuel to combustion chamber. Pump “P2" has a variable frequency drive which allows precise control of fuel mass flow to combustion chamber. It is assumed that combustion of delivered fuel mass flow is complete and instantaneous. It is also assumed that control of oxygen pump “P1" is linked to pump “P2" in a way that oxygen supply is in exact stoichiometric ratio needed for complete ethanol combustion. Therefore, oxygen pump “P1" dimensioning and control is not a part of this assignment. 
	Fuel pump “P2" speed determines the fuel combustion rate and influences evaporation and superheating of steam in heat exchanger “HE" and finally power P extracted at steam turbine “Turb".
	
	\subsection*{Fuel pump}
	
	Fuel pump “P2" is positive displacement pump with variable speed drive. It is used to transport fuel from fuel tank “T2" to combustion chamber “Comb". The total differential head a pump must generate is a sum of static head difference and frictional head losses: 
	
	\begin{equation}\label{eq:head_total}
		h_{tot} = h_{stat} + h_{fr}.
	\end{equation}
	
	There is no height difference between fuel tank “T2" and combustion chamber “Comb". Fuel is stored at pressure of 1 atm.
	Frictional head loss is calculated based on fuel volume flow rate:
	
	\begin{equation}\label{eq:head_loss}
		h_{fr} = k_{pipe} \cdot q_{v,fuel}^2,
	\end{equation}
	
	\noindent
	where
	
	\begin{itemize}
		\item $k_{pipe}$ is pipe friction coefficient $\left[ \frac{mh^2}{l^2} \right]$
		\item $q_{v,fuel}^2$ is fuel volume flow rate
	\end{itemize}

	\noindent
	It is assumed that positive displacement pump volume flow rate depends only on pump speed and does not depend on pump head. Pump slippage is neglected. Pump volume flow rate with respect to pump speed is defined:
	
	\begin{equation}\label{eq:pump_vol_flow}
		q_v = \frac{N}{N_{max}} q_{v,max},
	\end{equation}
	
	\noindent
	where
	
	\begin{itemize}
		\item $N$ is pump speed
		\item $N_{max}$ is maximum pump speed
		\item $q_{v,max}$ is maximum pump volume flow rate.
	\end{itemize}

	\noindent
	Pump head is limited to $h_{max}$. Pump speed is limited at given submarine depth such that pump total head does not exceed maximum head:
	
	\begin{equation}\label{eq:total_head_limit}
		h_{stat} + k_{pipe} \left( \frac{N}{N_{max}} q_{v,max} \right)^2 \leq h_{max}.
	\end{equation}
	
	\noindent
	Pump parameters are given in table \ref{tab:pump_params}.
	
	\subsection*{Compressor}
	
	Part of flue gas at heat exchanger exit after mixing at mixing chamber “Mix" is returned to combustion chamber “Comb" by compressor “Comp". This flue gas return should ensure that flue gas stream “a" always exits combustion chamber at temperature below $\theta_{fg,max}$. Compressor “Comp" can provide any flue gas mass flow rate between minimum and maximum flow rate defined for that compressor type. It can be assumed that sufficient mass of flue gases is always available for return at mixing region “Mix".
	
	Compressor is controlled with respect to fuel pump speed.  Dependence between pump speed and return flue gas flow rate provided by the compressor is linear. 
	
	For fully defining correlation between compressor and pump, compressor loads $L_{Comp}$ at two relative pump speed $N_{rel,1}$ and $N_{rel,2}$ have to be provided. Compressor load can have a value between zero and one and is defined as:
	
	\begin{equation}\label{eq:compressor_load}
		L_{Comp} = \frac{q_{m,ret} - q_{m,ret,min}}{q_{m,ret,max} - q_{m,ret,min}},
	\end{equation}
	
	\noindent
	where
	
	\begin{itemize}
		\item $q_{m,ret}$ is flue gas mass flow rate which compressor “Comp" returns to combustion chamber “Comb"
		\item $q_{m,ret,min}$ is minimal compressor mass flow rate
		\item $q_{m,ret,max}$ is maximal compressor mass flow rate	
	\end{itemize}

	\noindent
	Compressor load at some relative pump speed $N_{rel}$ is than:
	
	\begin{equation}\label{eq:compressor_load_rel}
		L_{Comp} = L_{Comp,1} + \frac{L_{Comp,2} - L_{Comp,1}}{N_{rel,2} - N_{rel,1}} \left( N_{rel} - N_{rel,1}\right)
	\end{equation}
	
	\noindent
	and is bounded between values 0 and 1.
	
	\subsection*{Condensate pump}
	
	Condensate pump “P3" is used for return of condensed steam at low condensation pressure $p_3$ condenser “Cond" outlet to heat exchanger “HE" at high evaporation pressure $p_4$.
	\textcolor{red}{(mislim da ovo možemo pojednostaviti na sljedeći način)}
	Neglecting a pump losses, condensate pump power is defined by:

	\begin{equation}\label{eq:pump_loss}
		P_{cp} = q_m(h_4 - h') = q_v(p_4 - p_3), 
	\end{equation}

	\noindent
	where
	
	\begin{itemize}
		\item 	$h'$ is enthalpy of dry saturated water at condensation pressure $p_3$
		\item $h_4$ is actual enthalpy after water compression from dry saturated state at pressure $p_3$ to pressure $p_4$
	\end{itemize}

	\subsection{Constants}
	
	\begin{itemize}
		\item $\theta_{fg,max} = 800 ^\circ C$
		\item $Re_{T_{crit}} = 3000$
		\item $Re_{L_{crit}} = 2300$
		\item $\alpha_{ev} = 3500 \frac{W}{m^2K}$
		\item $\alpha_{super} = 70 \frac{W}{m^2K}$
		\item $\theta_{fuel,in} = 10 ^\circ C$
		\item $\theta_{O_2,in} = 10 ^\circ C$
		\item $M_{ethanol} = 46 \frac{kg}{kmol}$
		\item $M_{CO_2} = 44 \frac{kg}{kmol}$
		\item $M_{H_2O} = 18 \frac{kg}{kmol}$
		\item $\Delta h_{md} = 1366940 \frac{kJ}{kmol}$
		\item $C_{mp,O_2}  = $
		\item $C_{mp,C_2H_6O} = $
		\item $\rho_{sea\ water} = 1000 \frac{kg}{m^3}$
		\item $\eta_{cond\ pump} = 0.75$
	\end{itemize}
		
\end{document}
