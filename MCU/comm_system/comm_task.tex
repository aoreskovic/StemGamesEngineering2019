Podmornica s broda prima koordinate koje definiraju njezinu putanju.
Koordinate su reprezentirane nizom ASCII znakova koji se šalju u paketima. Podaci se zatim moduliraju na QPSK nosioc središnje frekvencije 30 kHz i odašilju hidrofonom.

Na podmornici se prijemnik sastoji od prijemnog hidrofona, I/Q demodulatora i idealnog niskopropusnog filtera granične frekvencije fg=10kHz. Izlazi demodulatora dovode se na ADC-ove mikrokontrolera. Na mikrokontroler je spojena i vanjska jedinica koja generira impulse sinkronizirane s početkom svakog QPSK simbola. Ti impulsi su spojeni na IRQ pin mikrokontolera. Blok shema cijelog sustava zadana je slikom:
%TODO slika

Potrebno je implementirati sustav za demodulaciju i dekodiranje informacije iz ultrazvučnog signala. Prekidna rutina mora dohvatiti potreban broj uzoraka i spremiti ga u globalni FIFO buffer. Glavni program iz FIFO buffera dohvaća uzorke, obavlja demodulaciju i dekodiranje informacije. Primljene koordinate se trebaju ispisati na stdout u formatu:



## Dodatni zadatak

Uz informacije o koordinatama, brod odašilje i informacije o stanju mreže. Podaci su kodirani, a sastoje se od trenutne pozicije broda, točnog vremena i rutina koje podmornici omogućuju da predvidi rutu kojom će se kretati. Te dodatne informacije su slijedno raspodjeljene na 23 QPSK modulatora i multipleksirane, zajedno s osnovnim podacima iz prethodnog zadatka, u OFDM multipleks koji se sastoji od ukupno 34 podnosioca. Podnosioci su raspodjeljeni kao na slici
%TODO slika
P D D D P D D D P D D O P D D D P D D D P D D D P D D D P D D D P 
Neoznačeni podnosioci sadrže konstantan podatak.

Potrebno je modificirati kod iz glavnog dijela zadatka tako da se dekodira i dodatni set informacija sadržan u ostalim podnosiocima. Pretpostavite da je signal selektivno zagušen prolaskom kroz kanal nepoznate prijenosne funkcije.