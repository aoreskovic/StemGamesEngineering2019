\documentclass{article}
\usepackage{standalone}
\usepackage{fancyhdr}
\usepackage{graphicx}
\graphicspath{{Images/Crane/}}
\usepackage[e]{Template/gameshf}
\usepackage{import}
\usepackage{caption}
\usepackage{amsfonts, amsmath, amsthm}
\usepackage{makecell}

%opening
\title{STEM games 2019}
\author{Mentori}

\title{STEM Games Engineering Arena}
\date{}

\begin{document}
	\maketitle
	
	\section{Introduction}
	
	\noindent 
	\textbf{Friendly advice}: Read everything before solving the task. Task grading is automatic. Submission form for every task is described in task's readme file.
	
	\section{Tasks} \label{sec:tasks}
	
	The system for which you have to design a controller is the crane shown in the Figure \ref{fig:isometry}. The crane consists of a base, two booms(in some crane configurations \textit{Boom 1} is also called \textit{Mast}), two telescopic beams, three hydraulic cylinders, two electric motors and pulley with belt system. Masses of each part of the crane are listed in Table \ref{tab:crane_tab}.\textbf{ For the sake of simplicity and overall system dynamics, load has the mass $m_{load} = 800.59 kg$}.
	
	\begin{figure}[h!]
		\centering
		\includegraphics[width=\textwidth]{kran_teret_izometrija.jpg}
		\caption{Crane configuration, isometry.}
		\label{fig:isometry}
	\end{figure}
	
	\begin{center}
		\captionof{table}{Crane mass}\label{tab:crane_tab}
		\begin{tabular}{||c|| c || c|| c ||}
			\hline
			Part & Mass$[kg]$ & Part & Mass$[kg]$ \\
			\hline\hline
			Base + Base electric motor & 3794.43 & Rotation cylinder & 105.30\\ 
			\hline
			Boom 1 & 536.36 & Translation cylinder 1 & 199.42\\
			\hline
			Boom 2 & 807.88 & Translation cylinder 2 & 198.24\\
			\hline
			Telescope 1 & 664.18 & Pulley & 12.25\\
			\hline
			Telescope 2 & 611.33 & Pulley electric motor & 47.23\\
			\hline
		\end{tabular}
	\end{center}
	
	\subsection{Hydraulics system}
	
	Part of the crane is actuated using hydraulic system. The system, as we designed it, is shown in Figure \ref{fig:hydraulic} and we are aware it has some flaws. Your first task is to think about how you could redesign the system and improve its performance. Concentrate on what you can conclude with what we have given you, i.e. we ask you to propose a redesign of the schematics. You might want to postpone this task until you are done with other tasks and have a better understanding of the system you are working with.
	
	\begin{figure}[h!]
		\centering
		\includegraphics[width=0.75\textwidth]{hidraulika_shema.jpg}
		\caption{Hydraulic system schematics.}
		\label{fig:hydraulic}
	\end{figure}
	
	\subsection{Kinematics}
	
	Crane configuration and dimensions are shown in Figure \ref{fig:crane_side}, Figure \ref{fig:crane_top} and Figure \ref{fig:crane_back}. Figure \ref{fig:cylinder_conf} shows cylinder configuration with corresponding dimensions given in Table \ref{tab:cylinder_tab}. Load is shown in Figure \ref{fig:load}.
	
	\begin{figure}[h!]
		\centering
		\includegraphics[width=\textwidth]{kran_bokocrt.jpg}
		\caption{Crane configuration, side view.}
		\label{fig:crane_side}
	\end{figure}
	
	\begin{figure}[h!]
		\centering
		\includegraphics[width=\textwidth]{kran_tlocrt.jpg}
		\caption{Crane configuration, top view.}
		\label{fig:crane_top}
	\end{figure}
	
	\begin{figure}[h!]
		\centering
		\includegraphics[width=0.65\textwidth]{kran_nacrt.jpg}
		\caption{Crane configuration, back view.}
		\label{fig:crane_back}
	\end{figure}
	
	\noindent
	\textbf{Crane configuration has been sketched for $q_1 = 0 ^\circ$, $d_1 = 194mm$, $d_2 = 369mm$, $d_3 = 437.22mm$ and $q_2 = 0 ^\circ$. Base coordinate system is defined by red coordinate axes while red dot $L$ represents bottom of the load.} Limits of motor angles and cylinder offsets are given in Table \ref{tab:limits}.
	
	\begin{center}
		\captionof{table}{Actuator position limits}\label{tab:limits}
		\begin{tabular}{|| c || c c c c c ||}
			\hline
			Variable & $q_1$[^{\circ}] & $d_1$[m] & $d_2$[m] & $d_3$[m] &  $q_2$[^{\circ}]\\
			\hline\hline
			Minimum value & \times & 0 & 0 & 0 & 0 \\ 
			\hline
			Maximum value & \times & 0.8 & 3.5 & 3.5 & 20000 \\ 
			\hline
		\end{tabular}
	\end{center}
	
	\begin{figure}
		\centering
		\includegraphics[width=0.75\textwidth]{cilindar_shema.jpg}
		\caption{Cylinder configuration with dimensions.}
		\label{fig:cylinder_conf}
	\end{figure}
	
	\begin{center}
		\captionof{table}{Cylinder dimensions}\label{tab:cylinder_tab}
		\begin{tabular}{||c|| c c c || c c c ||}
			\hline
			Cylinder & D1[mm] & D2[mm] &  D3[mm] & H1[mm] & H2[mm] & H3[mm] \\
			\hline\hline
			Rotation & 180 & 200 & 50 & 30 & 820 & 20\\ 
			\hline
			Translation & 80 & 100 & 37.5 & 30 & 3520 & 20 \\
			\hline
		\end{tabular}
	\end{center}
	
	\begin{figure}[h!]
		\centering
		\includegraphics[width=0.8\textwidth]{teret.jpg}
		\caption{Load}
		\label{fig:load}
	\end{figure}
	
	\subsubsection{Direct}
	
	To be able to design a controller, first you have to understand  kinematics of the crane. Your task is to find the position of the load $L(x_L,y_L,z_L)$ for every combination of electric motor angles and cylinder offsets given in Table \ref{tab:direct}. Be careful calculating the result because \textbf{maximum offset} from correct solution we will tolerate is $\pm 0.25$ m.
	\noindent NOTE: Pulley is trickier than it looks.
	
	
	\begin{center}
		\captionof{table}{Direct kinematics configurations}\label{tab:direct}
		\begin{tabular}{|| c || c c c c c ||}
			\hline
			Configuration & $q_1$[^{\circ}] & $d_1$[m] & $d_2$[m] & $d_3$[m] &  $q_2$[^{\circ}]\\
			\hline\hline
			1. & 0 & 0.194 & 1 & 1 & 1500 \\ 
			\hline
			2. & 45 & 0.4 & 0 & 2 & 3600 \\
			\hline
			3. & 90 & 0.7 & 3.5  & 3.5 & 5000 \\
			\hline
		\end{tabular}
	\end{center}
	
	\noindent
	Information on how to submit your solution will be given to you in the \textit{readme} file in task folder. If you have any questions regarding submission of the task, ask your mentors.
	
	\subsubsection{Inverse}
	
	Next step is to be able to determine kinematic configuration in opposite direction. Based on load positions specified in Table \ref{tab:inverse}, determine the corresponding angles and offsets $(q_1, d_1, d_2, d_3, q_2)$ for every given configuration. It is often the case that inverse kinematics has multiple valid solutions, so any valid solution you give will be graded as successful. We will set the crane in configuration position you give us and check if it is inside the maximum offset range of $\pm 0.25$ m from reference point given in Table \ref{tab:inverse}.
	
	\begin{center}
		\label{tab:inverse}
		\captionof{table}{Inverse kinematics configurations}\label{tab:inverse}
		\begin{tabular}{|| c || c c c ||}
			\hline
			Configuration & $x_L$[m] & $y_L$[m] & $z_L$[m] \\
			\hline\hline
			1. & 8.0 & -5.0 & 0.0\\ 
			\hline
			2. & -7.5 & -5.0 & 7.5 \\
			\hline
			3. & 0.0 & 0.5 & 2.5 \\
			\hline
		\end{tabular}
	\end{center}
	
	\noindent
	Information on how to submit your solution will be given to you in the \textit{readme} file in task folder. If you have any questions regarding submission of the task, ask your mentors.
	
	\subsection{Model identification}
	
	Now when you understand kinematics of the crane, it would be useful to know its dynamic model.
	
	As mentioned before, the crane has 5 actuators. There is a DC motor which is turning the whole crane, one hydraulic actuator performing rotation of booms and telescopes (lifting), two hydraulic actuators used for extension of telescopes and finally a DC motor running belt and pulley system. To successfully drive the crane, you have to determine how it reacts to given inputs. In real systems, you often don't know how system will respond to your inputs, mostly because the model of the system is unknown. In many cases, there are some unknown parameters, order of model is unknown or maybe some parameters are not correct due to the age of system you are controlling, for example, old batteries, old motors, etc.
	
	For this task you will do exactly that! Some of the parameters of the system are unknown and there is no way to measure them so you have to identify models for all crane actuators.
	
	Crane is set to default start position ($q_1 = 0; \ d_1 = 0.194; \  d_2 = 0; \  d_3 = 0; \  q_2 = 0$) and it is ready for driving. Given the known parameters of actuators (Table ), your job is to determine how the actuator will respond. We give you the opportunity to test whatever input signals you want.
	Inputs for DC motors are in Volts [V] and for hydraulic actuators are valve opening in Meters[m]. Input signal limits are given in Table \ref{tab:input_limits}.  In task's \textit{readme} file you can find a way how to test your input signals and submit your results.
	
	\begin{center}
		\label{tab:inverse}
		\captionof{table}{Electric motor parameters}\label{tab:el_params}
		\begin{tabular}{|| c || c c c c c||}
			\hline
			Parameter & $R$[\Omega] & $L$[H] & $k$[V/s] & $J_{rot}$[kgm^2] & $n$\\
			\hline\hline
			Base electric motor &  &  &  & &\\ 
			\hline
			Pulley electric motor &  &  &  & & \\
			\hline
		\end{tabular}
	\end{center}
	
	\begin{center}
		
		\captionof{table}{Actuator input signal limits}\label{tab:input_limits}
		\begin{tabular}{|| c || c c c c c ||}
			\hline
			Actuator & \makecell{Base electric \\ motor [V]} & \makecell{Rotation \\ cylinder \\ valve[m]} & \makecell{Translation \\ cylinder 1 \\ valve[m]} & \makecell{Translation \\ cylinder 2 \\ valve[m]} &  \makecell{Pulley electric \\ motor [V]}\\
			\hline\hline
			\makecell{Minimum \\ value} & -250 & -0.005 & -0.005 & -0.005 & -180 \\ 
			\hline
			\makecell{Maximum\\ value} & 250 & 0.005 & 0.005 & 0.005 & 180 \\ 
			\hline
		\end{tabular}
	\end{center}
	
	We will test your solution with randomly generated test signals. To prevent you from sending us your test signal as an input for identification, every time you send signal for identification we will create new test signal and \textbf{erase all your previous results}. That means that after you decide that your test identification is good enough for you, you should not send another identification signal.
	As said before, test signals are generated randomly. In total, you will have 5 test cases.  Each test case will start in default start position . 
	Identification signal for every actuator is \textbf{step} function  (Heaviside function) with random step time in interval [0, $T_{sim}$] and with random value amplitude. Example of one test signal can be found in the task folder.
	
	\subsection{Crane driving}
	
	After creating both kinematic and dynamic models of the system, your task is to make it serve its purpose. Crane has to carry a specific cargo from the dock to the submarine cargo space. Desired positions are given in Table \ref{tab:inverse} and you have already calculated inverse dynamics for these coordinates. Crane has to place the load in these points in the order given in Table \ref{tab:trajectory}.
	
	\begin{center}
		\captionof{table}{Desired crane positions}\label{tab:trajectory}
		\begin{tabular}{|| c c c c c ||}
			\hline
			$P_1$ & $P_2$ & $P_3$ & $P_4$ & $P_5$\\
			\hline\hline
			3 & 1 & 3 & 2 & 3  \\ 
			\hline
		\end{tabular}
	\end{center}
	
	\noindent
	At each stop, the load has to stay steady for at least $t_{steady} = 5s$ before proceeding to the next one. "Stay steady" in this context means that the bottom point $L(x_L,y_L,z_L)$ of the load has to stay in the norm ball of $R = 0.25$ m around the reference point. Test is finished after the load has been in all requested positions or when the maximum simulation time $T_{sim} = 250s$ has passed. This is measured using test time $T_{test}$. Final performance cost $J_{total}$ is calculated using the following expression:
	
	\begin{equation} \label{eq:total_cost}
	J_{total} = \frac{5}{24}\sum_{i=1}^{5} D_i + \frac{75}{2} T_{total},
	\end{equation}
	
	\begin{equation} \label{eq:distance_cost}
	D_i = \left\{
	\begin{array}{ll}
	\sum_{t=T_{i-1}}^{T_{i}} ||L(t) - P_i||, &  P_i \textrm{ reached}, \\
	& \\
	10000, &  P_i \textrm{ not reached},\\
	\end{array} 
	\right.
	\end{equation}
	
	\noindent
	where $T_i$ is the time in which $t_{steady}$ has been reached in position $P_i$ and $T_0 = 0s$.
	
	\vspace{10pt}
	\noindent
	Your task is to design five controllers, one for every actuator. Discretization time for every controller is $T_s = 50ms$. Information on how to submit your solution will be given to you in the \textit{readme} file in task folder. If you have any questions regarding submission of the task, ask your mentors.
	
\end{document}

